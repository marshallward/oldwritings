After noticing Helmholtz's admirable discovery of the law of vortex motion in a
perfect liquid---that is, in a fluid perfectly destitute of viscosity (or fluid
friction)---the author said that this discovery inevitably suggests the idea
that Helmholtz's rings are the only true atoms. For the only pretext seeming to
justify the monstrous assumption of infinitely strong and infinitely rigid
pieces of matter, the existence of which is asserted as a probable hypothesis
by some of the greatest modern chemists in their rashly-worded introductory
statements, is that urged by Lucretius and adopted by Newton---that it seems
necessary to account for the unalterable distinguishing qualities of different
kinds of matter. But Helmholtz has proved an absolutely unalterable quality in
the motion of any portion of a perfect liquid in which the peculiar motion
which he calls ``Wirbelbewegung'' has been once created. This any portion of a
perfect liquid which has ``Wirbelbewegung'' has one recommendation of
Lucretius's atoms---infinitely perennial specific quality. To generate or to
destroy ``Wirbelbewegung'' in a perfect liquid can only be an act of creative
power. Lucretius's atom does not explain any of the properties of matter
without attributing them to the atom itself. Thus the ``clash of atoms,'' as it
has been well called, has ben invoked by his modern followers to account for
the elasticity of gases. Every other property of matter has similarly required
an assumption of specific forces pertaining to the atom. It is as easy (and as
improbable---not more so) to assume whatever specific forces may be required in
any portion of matter which possesses the ``Wirbelbewegung,'' as in a solid
indivisible piece of matter; and hence the Lucretius atom has no more {\it
prima facie\/} advantage over the Helmholtz atom. A magnificent display of
smoke-rings, which he recently had the pleasure of witnessing in Professor
Tait's lecture-room, diminished by one the number of assumptions required to
explain the properties of matter on the hypothesis that all bodies are composed
of vortex atoms in a perfect homogeneous liquid. Two smoke-rings were
frequently seen to bound obliquely from one another, shaking violently from the
effects of the shock. The result was very similar to that observable in two
large india-rubber rings striking one another in the air. The elasticity of
each smoke ring seemed no further from perfection than might be expected in a
solid india-rubberring of the same shape, from what we know of the viscosity of
india-rubber. Of course this kinetic elasticity of form is perfect elasticity
for vortex rings in a perfect liquid. It is at least as good a beginning as the
``clash of atoms'' to account for the elasticity of gases. Probably the
beautiful investigations of D. Bernoulli, Herapath, Joule, Kr\"onig, Clausius,
and Maxwell, on the various thermodynamic properties of gases, may have all the
positive assumptions they have been obliged to make, as to mutual forces
between two atoms and kinetic energy acquired by individual atoms or molecules,
satisfied by vortex rings, without requiring any other property in the matter
whose motion composes them than inertia and incompressible occupation of space.
A full mathematical investigation of the mutual action between two vortex rings
of any given magnitudes and velocities passing one another in any two lines, so
directed that they never come nearer one another than a large multiple of the
diameter of either, is a perfectly solvable mathematical problem; and the
novelty of the circumstances contemplated presents difficulties on an exciting
character. Its solution will become the foundation of the proposed new kinetic
theory of gases. They possibility of founding a theor of elastic solids and
liquids on the dynamics of more closely-packed vortex atoms may be reasonably
anticipated. It may be remarked in connexion with this anticipation, that the
mere title of Rankine's paper on ``Molecular Vortices,'' communicated to the
Royal Society of Edinburgh in 1849 and 1850, was a most suggestive step in
physical theory.

Diagrams and wire models were shown to the Society to illustrate knotted or
knitted vortex atoms, the endless variety of which is infinitely more than
sufficient to explain the varieties and allotropies of known simple bodies and
their mutual affinities. It is to be remarked that two ring atoms linked
together or one knotted in any manner with its ends meeting, constitute a
system which, however it may be altered in shape, can never deviate from its
own peculiarity of multiple continuity, it being impossible for the matter in
any line of vortex motion to go through the line of any other matter in such
motion or any other part of its own line. In fact, a closed line of vortex core
is literally indivisible by any action resulting from vortex motion.

The author called attention to a very important property of the vortex atom,
with reference to the now celebrated spectrum-analysis practically established
by the discoveries and labours of Kirchoff and Bunsen. The dynamical theory of
this subject, which Professor Stokes had taught to the author of the present
paper before September 1852, and which he has taught in his lectures in the
University of Glasgow from that time forward, required that the ultimate
constitution of simple bodies should have one or more fundamental periods of
vibration, as has a stringed instrument of one or more strings, or an elastic
solid consisting of one or more tuning-forks rigidly connected. To assume such
a property in the Lucretius atom, is at once to give it that very flexibility
and elasticity for the explanation of which, as exhibited in aggregate bodies,
the atomic constitution was originally assumed. If, then, the hypothesis of
atoms and vacuum imagined by Lucretius and his followers to be necessary to
account for the flexibility and compressibility of tangible solids and fluids
were really necessary, it would be necessary that the molecule of sodium, for
instance, should be not an atom, but a group of atoms with void space between
them. Such a molecule could not be strong and durable, and thus it loses the
one recommendation which has given it the degree of acceptance it has had among
philosophers; but, as the experiments shown to the Society illustrate, the
vortex atom has perfectly definite fundamental modes of vibration, depending
solely on that motion the existence of which constitutes it. The discovery of
these fundamental modes forms an intensely interesting problem of pure
mathematics. Even for a simple Helmholtz ring, the analytical difficulties
which it presents are of a very formidable character, but certainly far from
insuperable in the present state of mathematical science. THe author of the
present communication had not attempted, hitherto, to work it out except for an
infinitely long, straight, cylindrical vortex. For this case he was working out
solutions corresponding to every possible description of infinitesimal
vibration, and intended to include them in a mathematical paper which he hoped
soon to be able to communicate to the Royal Society. One very simple result
which he could now state is the following. Let such a vortex be given with its
section differing from exact circular figure by an infinitesimal harmonic
deviation of order $i$. This {\it form\/} will travel as waves round the axis
of the cylinder in the same direction as the vortex rotation, with an angular
velocity equal to $(i-1)/i$ of the angular velocity of the rotation. Hence, as
the number of crests is equal to $i$, for an harmonic deviation of order $i$
there are $i-1$ periods of vibration in the period of revolution of the vortex.
For the case $i=1$ there is no vibration, and the solution expresses merely an
infinitesimally displaced vortex with its circular form unchanged. The case
$i=2$ corresponds to elliptic deformation of the circular section; and for it
the period of vibration is, therefore, simply the period of
revolution\footnote{\dag}{Did Kelvin mean ``...simply half of the period of
revolution''? ---Ed.}. The results are, of course, applicable to the Helmholtz
ring when the diameter of the approximately circular section is small in
comparison with the diameter of the ring, as it is in the smoke-rings exhibited
to the Society. The lowest fundamental modes of the two kinds of transverse
vibrations of a ring, such as the vibrations that were seen in the experiments,
must be much graver than the elliptic vibration of section. It is probable that
the vibrations which constitute the incandescence of sodium-vapour are
analogous to those which the smoke-rings had exhibited; and it is therefore
probable that the period of each vortex rotation of the atoms of sodium-vapour
is much less than $1 \over 525$ of the millionth of the millionth of a second,
this being approximately the period of vibration of the yellow sodium light.
Further, inasmuch as this light consists of two sets of vibrations coexistent
in slightly different periods, equal approximately to the time just stated, and
of as nearly as can be perceived equal intensities, the sodium atom must have
two fundamental modes of vibration, having those for their respective periods,
and being about equally excitable by such forces as the atom experiences in the
incandescent vapour. This last condition renders it probable that the two
fundamental modes concerned are approximately similar (and not merely different
orders of different series chancing to concur very nearly in their periods of
vibration). In an approximately circular and uniform disk of elastic solid the
fundamental modes of transverse vibration, with nodal division into quadrants,
fulfil both the conditions. In an approximately circular and uniform ring of
elastic solid these conditions are fulfilled for the flexural vibrations in its
plane, and also in its transverse vibrations perpendicular to its own plane.
But the circular vortex ring, if created with one part somewhat thicker than
another, would not remain so, but would experience longitudinal vibrations
round its own circumference., and could not possibly have two fundamental modes
of vibration similar in character and approximately equal in period. The same
assertion may, it is probable\footnote{*}{{\it Note}, April 26, 1867.---The
author has seen reason for believing that the sodium characteristic might be
realized by a certain configuration of a single line of vortex core, to be
described in the mathematical paper which he intends to communicate to the
Society.}, be practically extended to any atom consisting of of a single vortex
ring, however involved, as illustrated by those of the models shown to the
Society which consisted of only a single wire knotted in various ways. It
seems, therefore, probable that the sodium atom may not consist of a single
vortex line; but it may very probably consist of two approximately equal vortex
rings passing through one another like two links of a chain. It is, however,
quite certain that a vapour consisting of such atoms, with proper volumes and
angular velocities in the two rings of each atom, would act precisely as
incandescent sodium-vapour acts---that is to say, would fulfil the ``spectrum
test'' for sodium.

The possible effect of change of temperature on the fundamental modes cannot be
pronounced upon without mathematical investigation not hitherto executed; and
therefore we cannot say that the dynamical explanation now suggested is
mathematically demonstrated so far as to include the very approximate identity
of the periods of the vibrating particles of the incandescent vapour with those
of their corresponding fundamental modes at the lower temperature at which the
vapour exhibits its remarkable absorbing-power for the sodium light.

A very remarkable discovery made by Helmholtz regarding the simple vortex ring
is that it always moves, relatively to the distant parts of the fluid, in a
direction perpendicular to its plane, towards the side towards which the
rotatory motion carries the inner parts of the ring. [INSERT PICTURE] The
determination of the velocity of this motion, even approximately, for rings of
which the sectional radius is small in comparison with the radius of the
circular axis, has presented mathematical difficulties which have not yet been
overcome\footnote{*}{See, however, note added to Professor Tait's translation
of Helmholtz's paper (REFERENCE 1), where the result [see {\it infra}, p. 67]
of a mathematical investigation which the author of the present communication
has recently succeeded in executing is given.}.

\bye
