\documentclass[letterpaper, 10pt]{report}

\usepackage{amsmath, amssymb, amsfonts}
\usepackage[dvips]{graphicx}

\newcommand{\pdiff}[2]{\frac{\partial #1}{\partial #2}}
\newcommand{\ddt}{\tfrac{\partial}{\partial t}}

\title{2D and Perspective Graphics on the Game Boy Advance}
\author{Marshall Ward}

\begin{document}

\maketitle

%%%%%%%%%%%%%%%%%%%%%%%%%%%%%%%%%%%%%%%%%%%%%%%%%%%%%%%%%%%%%%%%%%%%%%%%%%%%%%%
%%%%%%%%%%%%%%%%%%%%%%%%%%%%%%%%%%%%%%%%%%%%%%%%%%%%%%%%%%%%%%%%%%%%%%%%%%%%%%%
\chapter{Introduction}

Testing.

%%%%%%%%%%%%%%%%%%%%%%%%%%%%%%%%%%%%%%%%%%%%%%%%%%%%%%%%%%%%%%%%%%%%%%%%%%%%%%%
%%%%%%%%%%%%%%%%%%%%%%%%%%%%%%%%%%%%%%%%%%%%%%%%%%%%%%%%%%%%%%%%%%%%%%%%%%%%%%%

\chapter{Matrix Mathematics}

Many techniques in computer graphics can be described by elements known as \emph{matrices}. For example, one can associate the rotation of an object with a rotation matrix, which is applied to our coordinate system in some way, \emph{transforming} it from its original form to the new, rotated state. Other graphical processes such as translation, stretching, and shearing also have corresponding matrices. While the use of matrices requires a certain level of mathematical rigor, they also allow us to decompose a complicated scene into a series of basic matrices, which can then be combined through a simple (if sometimes tedious) procedure into a single transformation matrix.

%%%%%%%%%%%%%%%%%%%%%%%%%%%%%%%%%%%%%%%%%%%%%%%%%%%%%%%%%%%%%%%%%%%%%%%%%%%%%%%

\section{Linear systems}

Matrices are in fact operators that transform one linear system into another. To better understand this, consider the simplest linear system $y(x)$,
\begin{equation}
y = a x.
\end{equation}
From the perspective of linear algebra, this is not a linear system because it is the equation for a line, but because it has the following property: if there are two point $x_1$ and $x_2$ that have solutions $y_1$ and $y_2$, then the solution for the point $\alpha x_1 + \beta x_2$ is $\alpha y_1 + \beta y_2$. That is, if
\begin{align}
y_1 &= a x_1, \\
y_2 &= a x_2,
\end{align}
then it is easy to verify that
\begin{equation}
a \left( \alpha x_1 + \beta x_2 \right) = \alpha y_1 + \beta y_2.s
\end{equation}
It is also not difficult to show that the general equation for a line, $y(x) = a x + b$ is actually \emph{not} linear, in the linear algebraic sense!\footnote{The ``linear'' in linear algebra refers to this linearity property. The ``algebra'' refers to the use of addition and multiplication, or more generally any pair of operators with the same properties.}

While this is an interesting mathematical property, it also has important practical benefits. For example, if we know the solutions of a set of known quantities, say $x_1, x_2, \ldots$, and we can write an unknown $x_u$ in terms of these known quantities, then we can use the linear property to immediately write the solution of the unknown quantity, $y(x_u)$. In other words, any linear system can be decomposed into a set of potentially simpler problems, which can be solved individually and then used to build the solution of the more complicated problem.

For simple one-dimensional problems, this is usually not necessary. But when more than one variable is present, as in 2D and 3D graphics, such techniques become invaluable.

%%%%%%%%%%%%%%%%%%%%%%%%%%%%%%%%%%%%%%%%%%%%%%%%%%%%%%%%%%%%%%%%%%%%%%%%%%%%%%%
%%%%%%%%%%%%%%%%%%%%%%%%%%%%%%%%%%%%%%%%%%%%%%%%%%%%%%%%%%%%%%%%%%%%%%%%%%%%%%%
\end{document}