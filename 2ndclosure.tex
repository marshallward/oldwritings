\documentclass[landscape]{seminar}

\usepackage{amsmath, amssymb, amsfonts}

\usepackage{color}
\definecolor{darkblue}{rgb}{.0, .0, .6}

\usepackage[dvips]{graphicx}

\usepackage{hyperref}
\hypersetup{
  pdfmenubar=true,
  pdftoolbar=true,
  pdfpagemode={None}
}

\pagestyle{empty}

\renewcommand{\printlandscape}{\special{landscape}}
\newcommand{\pdiff}[2]{\frac{\partial #1}{\partial #2}}

\begin{document}

%%%%%%%%%%%%%%%%%%%%%%%%%%%%%%%%%%%%%%%%%%%%%%%%%%%%%%%%%%%%%%%%%%%%%%%%%%%%%%
\begin{slide}

\begin{center}
Review of Closure Modeling
\end{center}

\begin{itemize}

  \item We are principally interested in \emph{mean} quantities: $U_i$, $\Theta$, $P$

  $\Rightarrow 5$ unknowns of interest

\end{itemize}

Evolution equations:

\begin{equation*}
\frac{dU_j}{dx_j} = 0
\end{equation*}
\begin{equation*}
\pdiff{U_i}{t} + \pdiff{}{x_j}\left(U_i U_j + \overline{u_i u_j}\right) + \epsilon_{ijk} f_j U_k = -\pdiff{P}{x_i} - g_i \beta \Theta + \nu \nabla^2 U_i
\end{equation*}
\begin{equation*}
\pdiff{\Theta}{t} + \pdiff{}{x_j}\left(\Theta U_j + \overline{\theta u_j}\right) = \alpha \nabla^2 \Theta
\end{equation*}

\begin{itemize}
  \item 5 Evolution equations but 9 new unknowns: $\overline{u_i u_j}$, $\overline{\theta u_j}$

  $\Rightarrow$ System is not closed
\end{itemize}

\end{slide}
%%%%%%%%%%%%%%%%%%%%%%%%%%%%%%%%%%%%%%%%%%%%%%%%%%%%%%%%%%%%%%%%%%%%%%%%%%%%%%
\begin{slide}

\begin{itemize}
  \item We can model the unknown variables in terms of known variables; e.g. a down-gradient diffusions:
  \begin{center}
    $\overline{u_i u_j} = -K \pdiff{U_i}{x_j}$, \ \  $\overline{\theta u_j} = -k_T \pdiff{\Theta}{x_j}$
  \end{center}

  This is a \emph{first order} closure, solving for the first moments

  \item Instead we can determine evolution equations for the second moments $\overline{u_i u_j}$, etc., and hope that it improves our accuracy

\end{itemize}

\end{slide}
%%%%%%%%%%%%%%%%%%%%%%%%%%%%%%%%%%%%%%%%%%%%%%%%%%%%%%%%%%%%%%%%%%%%%%%%%%%%%%
\begin{slide}

\begin{center}
Second Moment Evolution Equations
\end{center}

\footnotesize

\[
\begin{split}
\pdiff{}{t}(\overline{u_i u_j}) &+ \pdiff{}{x_k}\left[\overline{u_i u_j} U_k + \overline{u_i u_j u_k} - \nu \pdiff{}{x_k}\overline{u_i u_j}\right] + f_k \left[\epsilon_{j k l} \overline{u_l u_i} + \epsilon_{i k l} \overline{u_i u_j} \right]\\
   & = - \left[\pdiff{}{x_j} \overline{p u_i} + \pdiff{}{x_i} \overline{p u_j} \right] - \left[\overline{u_i u_k} \pdiff{U_j}{x_k} + \overline{u_j u_k} \pdiff{U_i}{x_k} \right] \\
   & - \beta \left[g_j \overline{u_i \theta} + g_i \overline{u_j \theta} \right] + \overline{p \left( \pdiff{u_i}{x_j} + \pdiff{u_j}{x_i}\right)} - 2 \nu \pdiff{u_i}{x_k} \pdiff{u_j}{x_k}\\
\end{split}
\]

\[
\begin{split}
\pdiff{}{t}(\overline{\theta u_j}) &+ \pdiff{}{x_k} \left[U_k \overline{u_j \theta} + \overline{u_j u_j \theta} - \alpha \overline{u_j \pdiff{\theta}{x_k}} - \nu \overline{\theta \pdiff{u_j}{x_k}} \right] + f_k \epsilon_{jkl}\overline{u_l \theta} + \pdiff{}{x_j}\overline{p \theta}\\
  &  = - \overline{u_j u_k} \pdiff{\Theta}{x_k} - \overline{\theta u_k} \pdiff{U_j}{x_k} - \beta g_j \overline{\theta^2} + \overline{p \pdiff{\theta}{x_j}} - (\alpha + \nu) \overline{\pdiff{u_j}{x_k}\pdiff{\theta}{x_k}} \\
\end{split}
\]

\[
\pdiff{}{t}\overline{\theta^2} + \pdiff{}{x_k} \left[U_k \overline{\theta^2} + \overline{u_k \theta^2} - \alpha \pdiff{\overline{\theta^2}}{x_k} \right] = - 2 \overline{u_k \theta} \pdiff{\Theta}{x_k} - 2 \alpha \overline{\pdiff{\theta}{x_k} \pdiff{\theta}{x_k}}
\]

\normalsize

\end{slide}
%%%%%%%%%%%%%%%%%%%%%%%%%%%%%%%%%%%%%%%%%%%%%%%%%%%%%%%%%%%%%%%%%%%%%%%%%%%%%%
\begin{slide}

\begin{itemize}

  \item Same problem: Expressions for third moments are needed to determine second moment evolution

  \item \emph{Second Order Closure}: Model the third moments in terms of lower moments

  \item By physical and dimensional arguments, relate them to a \emph{kinetic energy scale} $q^2$, and a \emph{length scale} $l$, which also gives us a time scale $t_e = l / q$

\begin{center}
Examples:
\end{center}

$2 \nu \overline{\pdiff{u_i}{x_k} \pdiff{u_j}{x_k}} = \frac{2}{3} \frac{q^3}{B_1 l} \delta_{ij}$, \ \ \ $(\alpha + \nu) \overline{\pdiff{u_j}{x_k} \pdiff{\theta}{x_k} } = 0$,

\bigskip

$2 \alpha \overline{\pdiff{\theta}{x_k} \pdiff{\theta}{x_k} } = 2 \frac{q}{B_2 l} \overline{\theta^2}$, \ \ \ \ \ldots

\end{itemize}

\end{slide}
%%%%%%%%%%%%%%%%%%%%%%%%%%%%%%%%%%%%%%%%%%%%%%%%%%%%%%%%%%%%%%%%%%%%%%%%%%%%%%
\begin{slide}

\begin{itemize}

  \item First moments ($U_i$, $\Theta$, $P$) depend on themselves and second moments

  \item Second moments ($\overline{u_i u_j}$, $\overline{u_i \theta}$, $\overline{\theta^2}$) depend on first moments, second moments, and $q$ and $l$

  \item Equations for $q$ and $l$, or any two combinations of them, such as $q^2$ and $\epsilon = q^3/B_1 l$, will close our system of equations

  \item The \emph{kinetic energy equation} for $q^2$ is well established:
\[
\frac{D q^2}{Dt} - \pdiff{}{z} \left[q l S_q \pdiff{}{z}\left(q^2\right)\right] = 2(P + B - \epsilon)
\]

  \item The \emph{length scale equation}, or energy dissipation ($\epsilon$) equation, or whatever you choose, however, is not so well established

\end{itemize}

\end{slide}
%%%%%%%%%%%%%%%%%%%%%%%%%%%%%%%%%%%%%%%%%%%%%%%%%%%%%%%%%%%%%%%%%%%%%%%%%%%%%%

\end{document}