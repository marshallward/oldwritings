\documentclass[letterpaper, 11pt, onecolumn]{article}

\usepackage{amsmath, amssymb, amsfonts, bm}
\usepackage{newapa, natbib}
\usepackage[dvips]{graphicx}

%\usepackage{hyperref}
%\hypersetup{ % pdfmenubar=true, % pdftoolbar=true, %pdfpagemode={None} %}

%\pagestyle{empty}

\newcommand{\pdiff}[2]{\frac{\partial #1}{\partial #2}}
\newcommand{\degree}{^{\circ}} \newcommand{\abs}[1]{\lvert#1\rvert}
\newcommand{\energy}[1]{$#1 \times 10^{-3} \text{ cm}^2 \text{ s}^{-3}$}
\newcommand{\degC}{\degree\text{C}}

%%%%%%%%%%%%%%%%%%%%%%%%%%%%%%%%%%%%%%%%%%%%%%%%%%%%%%%%%%%%%%%%%%%%%%%%%%%%%%

\title{Tropical Instability Waves}

\author{Marshall Ward}

%%%%%%%%%%%%%%%%%%%%%%%%%%%%%%%%%%%%%%%%%%%%%%%%%%%%%%%%%%%%%%%%%%%%%%%%%%%%%%

\begin{document}

\maketitle

\section{Introduction}

Tropical instability waves (TIWs) are a particular set of phenomena that have
been observed over the last 25 years in the equatorial oceans. These
disturbances were originally associated with spatial variations of sea surface
temperature in the eastern Pacific, made visible by the strong gradients within
this region, and the resulting patterns would propagate westward with a
frequency of roughly 20--30 days. Such waves cannot be a part of the
traditional neutral spectrum of equatorial waves, for only the
eastward-traveling Kelvin and Yanai waves can exist within this frequency band.
However, just as these observations were first being conducted, there was also
newfound theoretical evidence showing that the mean equatorial currents were
unstable, and that the most unstable modes possessed wavelengths and
frequencies that were in agreement with those early observations. The
coinciding of these two discoveries led to a designation of the phenomenon as
tropical instability waves. Although the mechanism suggested by these early
studies is no longer regarded as correct, it is generally accepted that TIWS
are the result of an instability within the equatorial oceans, and is at least
in part due to equatorial current instabilities.

In this paper, we review some of the more recent observational evidence of
TIWs. We focus on two main studies: the first, by \citet{Chelton++:2000}, is a
modern treatment of the earliest evidence of TIWs, namely their SST profiles.
Only in the last few years have satellites with microwave imaging been
introduced, and their ability to penetrate cloud cover has vastly improved our
understanding of these patterns. The second study to be presented is by
\citet{Qiao-Weisberg:1998}, who have performed a detailed analysis of the eddy
energetics within the equator. For even though most would agree that TIWs are
caused by an instability of the mean state, the actual mechanism behind their
development is still not clear. By explicitly calculating the various sources
of eddy kinetic energy, it is hoped that we will gain a better understanding of
the underlying dynamics behind TIW generation and propagation.


\section{Observational Foundations}

The existence of TIWs was first surmised from the early infrared observations
of sea surface temperature (SST) over the Pacific \citep{Legeckis:1977},
followed by similar observations over the Atlantic
\citep{Legeckis-Reverdin:1987}. The most prominent measurements of TIWs have
been along the intense northern SST front of the eastern Pacific cold tongue.
The early measurements of Legeckis revealed a cusp-like pattern along this
front with a wavelength of about 1000~km and period of about 25~days, as well
as a westward phase speed of roughly 0.5~m/s. Subsequent measurements over the
years have largely confirmed these estimates.

Before 1997, measurements of the northern SST front were complicated by cloud
coverage over the tropics, which acts to impede visible and infrared radiation
from orbiting satellites. But with the launching of the Tropical Rainfall
Measuring Mission (TRMM) Microwave Imager (TMI) on November 1997, it is now
possible to use the microwave spectrum to determine tropical SST profiles.
Because such signals are transparent to clouds, measurements can now be made in
nearly all weather conditions, with the exception of regions of persistent
rainfall which tend to contaminate such observations. Since the rainfall is
both sporadic (on large timescales) and easily identifiable, it does not
seriously hinder such observations. The coverage of the TMI exceeds 90\% over
an area from $20\degree\text{S}$ to $20\degree\text{N}$ for a three-day
composite average, so we should be able to construct a reasonably accurate time
series for processes that occur on timescales of a week or more, such as TIWs.

\subsection{The Equatorial Pacific Cold Tongue}

%\begin{figure} %\begin{center} %\includegraphics{coldtongue.ps}
%\caption{Test}\label{Fig:ColdTongue} %\end{center} %\end{figure}

Since the presence of TIWs in the Pacific is strongly associated with the
equatorial cold tongue, we first present some of the basic properties of this
phenomenon. The tongue itself is a long band of cold water that extends from
South America to deep within the central Pacific. By following the tongue
westward, the latitude of the coldest water moves towards the north from about
$2\degree\text{S}$ at $85\degree\text{W}$ to the equator near
$130\degree\text{W}$, so that it is centered at about $1\degree\text{S}$. There
are strong SST fronts to the north and south across most of the tongue, with
the northern front tending to be located around $2\degree\text{N}$ and the less
well-defined southern front being pushed towards the equator as we follow the
tongue westward. While the tongue is typified by such fronts, the eastern cold
water turns southward at $100\degree\text{W}$, causing the southern front to
become effectively nonexistent. The northern front similarly dips southward in
this region, crossing the equator at $90\degree\text{S}$.

The intensity and spatial extent of the tongue changes dramatically on both
seasonal and interannual timescales. Although the tongue is always present in
some form, the SST gradients are most intense between the months of July and
December for a typical year. During this period, the northern front has a
gradient of roughly $1\degC$ per degree of latitude and the southern front is
significantly weaker, at about $0.5\degC$ per degree latitude. During an El
Ni\~no event, the warming of the east Pacific will consequently induce a net
warming of the tongue and a weakening of the fronts. Similarly, the large scale
cooling associated with a La Ni\~na event will cause the tongue to become even
more pronounced, strengthening the SST gradients along the fronts.

The signature of a TIW is the periodic disturbance of SST along these fronts,
and since the strongest tropical SST gradients are observed along the northern
fronts of the cold tongue, this is also where we observe the most intense TIW
activity. Moderately strong waves can be also seen along the southern front of
the cold tongue. A La Ni\~na event, which causes a further increase of SST
gradients, will summarily lead to an even greater intensification of TIWs. A
similar argument explains why an El Ni\~no event will tend to suppress TIW
activity.

As for the other tropical oceans of the world, there is a modest cold tongue
across the Atlantic which possesses a definite (although significantly weaker)
northern front but lacks a well-defined southern front. TIWs have been observed
along this northern front in the Atlantic. SST gradients in the Indian Ocean
have been too weak to resolve at this time, and hence no TIWs have been
observed in this part of the world.

\subsection{Evolution of TIWs}

To better understand the dynamics of these waves, we now describe the evolution
of a TIW event as reported by \citet{Chelton++:2000}. During the 1997--1998 El
Ni\~no, the SST fronts of the cold tongue were not visible, and no TIWs were
observed. On 13 May 1998, cold waters began to reappear within the tongue, east
of $140\degree\text{W}$, which signaled the end of the El Ni\~no event. Within
a week, the cold tongue intensified and stretched westward to the date line,
and almost immediately thereafter the cusp-like signature of a TIW was
observable along the northern front.

These cusps began as relatively small perturbations near the Galapagos at
$92\degree\text{W}$. They grew rapidly as they propagated westward, so that by
the time they reached about $110\degree\text{W}$, the SST fronts were being
meridionally displaced by several degrees of latitude. Many of the northern
cusps, which themselves pointed northward, began to eject clockwise swirls of
cold water, which are likely candidates for the anticyclonic eddies observed in
earlier measurements of drifter trajectories \citep{Hansen-Paul:1984}.

As we move deeper into the TIW season, similar cusps began to appear along the
southern SST front, and these became larger as the season progressed. By the
middle of October 1998, TIWs were highly developed along both fronts and the
wavetrains would propagate as far as $160\degree\text{E}$, with seemingly
identical phase speeds along each side of the tongue. At maximal extent, the
northern cusps were approximately 50\% larger than the southern cusps. The TIWs
and the fronts of the cold tongue remained well-defined until February 1999.
The phase relationship of the north and south TIWs is difficult to establish,
since the cusps become increasingly distorted as they protrude farther from the
equator. But the generally sinuous profile of the cold tongue suggests a
roughly antisymmetric relationship between the waves.

% This is apparently suggested by a theory of Yu++, 1995

The Atlantic was also monitored during this interval. A similar series of
westward-propagating cusps was observed at $1\degree\text{N}$, along the SST
front of its markedly weaker cold tongue. These waves appeared almost
immediately after the formation of this equatorial cold water on 26 April 1998.
In addition to having a cusp amplitude much weaker than those associated with
Pacific TIWs, they propagated at a reduced speed of about 0.3~m/s. After August
1998, the SST gradient weakened and it became much more difficult to track the
Atlantic TIWs.

Overall, the study by \citet{Chelton++:2000} established that TIWs are present
across the entire northern front of the Pacific cold tongue, from the coast of
South America to the western tip of the tongue at $160\degree\text{E}$, and may
even extend across the entire Pacific, although this cannot be determined by
the TMI due to the lack of strong SST gradients in the west. The existence of
TIWs along the southern front was also established. The microwave-based
measurements allowed for the compilation of uninterrupted sets of 50-day SST
profiles, which were used via Radon transforms to compute a phase speed of
0.53~m/s for TIWs along the northern front and 0.50~m/s along the southern
front. Atlantic TIW speeds were estimated at 0.31~m/s.

In addition to clarifying earlier studies on TIWs, they not only confirmed the
existence of the southern waves, but also found that they became pronounced
much later than their northern counterparts, and were actually persistent for a
month longer. Also, although the roughly 0.5~m/s TIWs are clearly dominant,
faster signals of lower amplitude were also observed.

Finally, there is an unambiguous correlation of TIW intensity with the presence
of mean SST gradients, but these gradients may not play a significant role in
the dynamics, and instead merely serve to highlight their existence. The
establishment of TIWs in regions of weak SST would require the measurement of
dynamical fields such as sea level or meridional velocity under similar
conditions, taken over regions such as the western Pacific.

\section{Energetics of TIWs}

Since the discovery of TIWs, a variety of dynamical theories on their origins
have been proposed. The earliest theory, developed in conjunction with their
discovery, suggested that they were due to barotropic instabilities caused by
the intense shear between the South Equatorial Current (SEC) and the North
Equatorial Countercurrent (NECC) \citep{Philander:1976, Philander:1978}. The
role of zonal equatorial currents was later well established in the stability
analysis of \citet{Cox:1980}. \citet{McCreary-Yu:1992} and
\citet{Yu-McCreary-Proehl:1995} clarified the importance of the cyclonic shear
between the SEC and Equatorial Undercurrent (EUC), and demonstrated how the
strong SEC-NECC shear may in fact play little if any role in the generation of
instabilities. They also proposed the existence of a baroclinic frontal
instability across the northern SST front, and claim that this is an essential
part of the TIW dynamics. \citet{Proehl:1996} relates the presence of TIWs to
the existence of critical layers and overreflection. Since both barotropic and
baroclinic processes are involved, he suggests that such a distinction does not
provide a coherent description of the instability. More recently,
\citet{Masina-Philander-Bush:1999} have presented a model which demonstrates
that baroclinic instabilities along the SST front trigger TIW formations, which
then transport energy southward and induce barotropic instabilities near the
equator. The phase locking of these two processes leads to TIW propagation.

While there is currently no definite consensus on the underlying dynamics of
TIW formation, observational evidence seems to be in favor of a barotropic
instability between the SEC and EUC. This is most evident in the recent study
of \citet{Qiao-Weisberg:1998}, who were able to calculate the relevant
perturbation energy fluxes during a TIW season. Data was collected from the
Tropical Instabilite Wave Experiment (TIWE), a crossbar network of five buoys
located on the equator at $140\degree\text{W}$, as well as the Tropical Ocean
Global Atmosphere Tropical Atmosphere Ocean (TOGA TAO) array.

\subsection{Eddy energetics} We now present a summary of the results by Qiao
and Weisberg.

To do this, we shall make a number of assumptions in our analysis of the
energetics. We presume the existence of mean flows $U$, $V$, $W$, but will only
allow variations in $U$ and $V$. Additionally, we assume a hydrostatic balance
and that the lateral variations of the isopycnals are small. Specifically, we
require that $\abs{\nabla \overline{\rho} / \overline{\rho}} \ll 1$ and
$\abs{\nabla \overline{\rho}_z / \overline{\rho}_z} \ll 1$. We shall also
assume that we have a Boussinesq-like approximation
$\abs{\rho'/\overline{\rho}} \ll 1$, all of which are wholly applicable to the
ocean. Frictional stress has also been neglected.\footnote{There are a number
of other implicit assumptions in Qiao and Weisberg's analysis. For example, the
term $\frac{1}{\overline{\rho}} \langle \rho' \bm{u}' \cdot \nabla p' \rangle$
has been wholly neglected from their analysis. But being a third moment
correlation, it is presumably small and not relevant to the analysis.}

The eddy (or perturbation) kinetic energy is $\text{PKE} = \frac{1}{2}
\overline{\rho} \left(u'^2 + v'^2\right)$, where the vertical flow has been
neglected. Under the assumptions above, the PKE evolves according to the
equation \begin{subequations}\label{PKE-eqn} \begin{align}
  \left(\text{PKE}\right)_t =& -U \left(\text{PKE}\right)_x -V
\left(\text{PKE}\right)_y -W \left(\text{PKE}\right)_z \label{PKE-MeanAdv}\\
  & - \langle u' \left(\text{PKE}\right)_x \rangle - \langle v'
\left(\text{PKE}\right)_y \rangle - \langle w' \left(\text{PKE}\right)_z
\rangle \label{PKE-EddyAdv} \\
  & - \langle u' u'\rangle U_x - \langle u' v' \rangle \left(U_y + V_x\right) -
\langle v' v' \rangle V_y - \langle u' w' \rangle U_z - \langle v' w' \rangle
V_z \label {PKE-MeanGen}\\
  & - \langle p'_x u' \rangle - \langle p'_y v' \rangle - \langle p'_z w'
\rangle \label{PKE-Work} \\
  & - g \langle \rho' w' \rangle \label{PKE-PEConv}. \end{align}
\end{subequations} The terms \eqref{PKE-MeanAdv} and \eqref{PKE-EddyAdv}
describe the advection of eddy kinetic energy by the mean and perturbation
flows, respectively. The terms \eqref{PKE-MeanGen} describe the barotropic
conversion of energy from the mean flow to the eddies. The terms
\eqref{PKE-Work} represents the work done to deform parcels by the perturbation
pressure field. Finally, term \eqref{PKE-PEConv} represents a conversion from
potential energy against the (hydrostatic) gravity field. Such conversions
across lateral isopycnal variations have been neglected.

To better understand the conversion terms, consider the following idealized
situation. If we have a mean flow of constant vertical shear $U = \Lambda z$,
with faster flows above, then the only barotropic conversion term from
\eqref{PKE-MeanGen} is $-\overline{\rho} \Lambda \langle u' w' \rangle$. Now
suppose that upward flows coincide with eastward flows and that there is a
similar correlation between downward and westward flows, so that $\langle u' w'
\rangle$ is everywhere positive. Then eastward momentum will tend to be carried
upward with westward momentum carried downward, and the flow will thereby tend
towards an organized shear of greater intensity. In this manner, the mean
kinetic energy would tend to increase while the PKE decreases, which is
consistent with our expression. If $u'$ and $w'$ are anticorrelated so that
$\langle u' w' \rangle < 0$, then eddy momentum would be transferred against
the shear and would tend to produce a more disorganized flow, and therefore an
increase of PKE.

As for the potential energy exchange \eqref{PKE-PEConv}, we associate heavier
water with $\rho' > 0$ and lighter water with $\rho' < 0$. Then if we are in a
situation where heavy parcels are rising, \emph{i.e.} $w' > 0$, while lighter
water is sinking, $w' < 0$, then the correlation $\langle \rho' w' \rangle$
will tend to be positive, and therefore we will also tend to produce eddy
potential energy (PPE) at the cost of PKE. The reverse situation will therefore
tend to produce PKE at the cost of PPE.

The eddy potential energy $\text{PPE} = \frac{1}{2} g \rho'^2 /
\left(-\overline{\rho}_z\right)$ obeys the equation
\begin{subequations}\label{PPE-eqn} \begin{align}
  \left(\text{PPE}\right)_t =& -U \left(\text{PPE}\right)_x - V
\left(\text{PPE}\right)_y -W \left(\text{PPE}\right)_z \label{PPE-MeanAdv}\\
  & - \langle u' \left(\text{PPE}\right)_x \rangle - \langle v'
\left(\text{PPE}\right)_y \rangle - \langle w' \left(\text{PPE}\right)_z
\rangle \label{PPE-EddyAdv} \\
  & + g \left(\overline{\rho}_x \langle \rho' u' \rangle + \overline{\rho}_y
\langle \rho' v' \rangle \right) / \overline{\rho}_z \label{PPE-LatGen} \\
  & + g \langle \rho' w' \rangle \label{PPE-VertGen}. \end{align}
\end{subequations} The terms \eqref{PPE-MeanAdv} and \eqref{PPE-EddyAdv}
describe advection of eddy potential energy by the mean and eddy flows. Terms
\eqref{PPE-LatGen} and \eqref{PPE-VertGen} describe a baroclinic extraction of
potential energy from the mean density profile, where \eqref{PPE-LatGen} is due
to lateral flows while \eqref{PPE-VertGen} is due to vertical flow.

The baroclinic PPE conversion terms can be understood in a similar manner as
the PKE. If a parcel moves across an isopycnal surface, then it must undergo a
change in potential energy. For example, as in the PKE equation, the term
\eqref{PPE-VertGen} illustrates how the rising of heavy water and sinking of
light water, associated with $\langle \rho' w' \rangle > 0$, will tend to
increase the available potential energy, which is consistent with our
expression. This term has the opposite sign of \eqref{PKE-PEConv} in the PKE
equation because it represents a direct exchange of kinetic and potential
energy. A similar effect occurs for horizontal flows, except that the intensity
is adjusted by terms such as $\overline{\rho}_x / \overline{\rho}_z \sim \delta
z / \delta x$, the isopycnal slope. We expect such exchanges to be small
however, and are not even represented in our PKE dynamics.

\subsection{Observations of Energetic Fluxes}

Through the use of TIWE and TOGA TAO data, Qiao and Weisberg were able to
estimate each term in the PKE and PPE equations, and could determine the local
exchanges of kinetic and potential energy, in an attempt to diagnose the
instability process which produces TIWs. Their focus is on a TIW event which
occurred over 1990, using data taken at $140\degree\text{W}$ and within $\pm
0.5\degree$ of the equator. A general survey of the measurements will show that
a region of major energy activity is one where the changes in energy per unit
mass exceed \energy{0.5}.

We first consider the perturbation kinetic energy. In \eqref{PKE-MeanAdv}, the
zonal and meridional advection of PKE by the mean flows are both found to be
significant during periods of wave activity, exceeding \energy{2}, and confined
to the upper $110 \text{ m}$. For the meridional patterns, there is a
divergence of PKE at the surface and a convergence at the subsurface, due to
divergent Ekman pumping, caused by increase of $\beta$, and the corresponding
convergence below. The vertical advection is negligible. When combining these
processes, the zonal and meridional advections are approximately balanced, with
total advections of at most \energy{1.5}, a significant reduction compared to
the individual components, so that advection alone cannot be responsible for a
buildup of PKE.

% Maybe omit this stuff about Ekman divergence. I sort of understand atop, but
not necessarily the subsurface stuff

For the perturbation PKE advection, most of these triple correlations are not
of much importance, rarely exceeding \energy{0.5}, with the exception of
meridional advection, $-\langle \left(u'^2 + v'^2\right) v' \rangle / 2$, which
may be as large as \energy{2.5}. The term varies greatly, but is overall
negative and tends to weaken the TIWs.

The dominant generation of PKE is due to the exchange between eddies and the
mean flow. The zonal shear sources $-\langle u' u' \rangle U_x$ and $-\langle
u' v' \rangle V_x$ are found to have relatively small magnitudes (less than
\energy{0.5}) and are not a significant source of energy. The term $-\langle u'
v' \rangle U_y$ appears to be the most significant source of PKE, with positive
values as large as \energy{2} observed between July and August above the EUC
core, suggesting that the TIWs are extracting energy from the meridional shear
between the zonal SEC and EUC flows.

Also of importance is the term $-\langle v' v' \rangle V_y$, which reflects the
mean meridional flow convergence and divergence since $\langle v' v' \rangle$
is positive definite. We observe positive values on the surface and negative
subsurface values, as expected. When we consider the sum of the meridional
shear generation terms, we find that on the equator and to its south, there are
negative values and therefore these regions act as a sink for PKE. In contrast
to this, the north tends to have predominantly positive values, largely due to
$U_y$, which produce the energy necessary for PKEs. These patterns essentially
coincide with the wave season.

As for the vertical shear production terms $-\langle u' w' \rangle U_z$ and
$-\langle v' w' \rangle V_z$, the $U_z$ shear tends to reverse sign across the
EUC core, with large negative values of about \energy{2} above $110 \text{ m}$,
but small positive values below. Hence, the $U_z$ shear principally acts as a
sink of PKE. So to summarize the effects of shear, the meridional shear $U_y$
to the north of the equator is the principal source of PKE, while $U_y$, $V_y$
and $U_z$ shears tend to reduce PKE at and south of the equator.

The pressure terms \eqref{PKE-Work} are the most difficult to evaluate, most
notably because there is no definite way to evaluate $-\langle p'_x u' \rangle$
from TIWE data. The meridional pressure work $-\langle p'_y v' \rangle$ is
found to be quite large, of about \energy{4.5}. While earlier observational
evidence seems to suggest that these two lateral work terms should balance, one
cannot realistically estimate their effects. But given that such terms require
an external forcing to be nonzero when integrated over a domain, they are most
likely not involved in TIW generation.

The last term of importance is the PKE-PPE exchange term, $-g \langle \rho' w'
\rangle$. If the perturbation density field is hydrostatic (\emph{i.e.}, $p'_z
= - g \rho'$), then this will precisely balance the work term $-\langle p'_z w'
\rangle$. In any case, this value is found to be positive for most of the TIW
season, and strongly negative following this season at around February. THe 
positive correlation shows an exchange from PKE to PPE.

We shall now briefly summarize the analysis of the PPE equation. In general,
most of the terms are not significant. Only mean meridional and vertical
advection can be estimated, and both are less than $10^{-3} \text{ cm}^2 \text{
s}^{-3}$. We then extrapolate that the same is true for zonal flows, and that
this is not a dominant part of PPE balance. The baroclinic conversion, which
could have the greatest effect on energetics, turns out to in fact be very
small, less than $10^{-3} \text{ cm}^2 \text{ s}^{-3}$. Again, only meridional
fluxes could be measured, but since we expect $\overline{\rho}_y >
\overline{\rho}_x$, we can extrapolate to both cases. We emphasize however that
these measurements were made near the equator, and not along the cold tongue
front. The only term of dominance is the PKE-PPE exchange, $g \langle \rho' w'
\rangle$, which has already been described. Therefore, we see that the change
in PPE along the equator is most likely due to only the PKE-PPE exchange.

We therefore conclude that along the equator, it is the meridional shear of
zonal flows $U_y$ that is the principal source of instability. We also see that
PKE sinks arise from $U_y$, $V_y$ and $U_z$, as well as the perturbative PKE
flux. There is also an exchange of PKE to PPE.

\subsection{Discussion of Energetics}

Because of the dominance of $-\langle u' v' \rangle U_y$ above the EUC core, we
conclude that a potential source of TIW energy is from the strong subsurface
shear between the SEC and EUC, and occurs just north of the equator. After the
initial growth, the waves seem to be maintained by the $U_y$ and $V_y$ shears
to the north, while the $U_y$, $V_y$ and $U_z$ shears tend to weaken the waves
and return the energy to the mean flow. The meridional perturbative PKE flux
$-\langle \left(u'^2 + v'^2\right) v' \rangle / 2$ further weakens the waves in
the south by creating a meridional dispersive effect. The meridional work
$-\langle p'_y v' \rangle$ also weakens the waves, but may be counteracted by
the zonal work $-\langle p'_x u' \rangle$. For the vertical flux, we expect a
generation of PKE near $180 \text{ m}$ through PKE-PPE exchange, while the
vertical work $-\langle p'_z w' \rangle$ then acts to carry the energy upward.

\section{Conclusions}

We have presented some of the most striking evidence for the existence of TIWs
by describing one of the more recent observations of SST profiles by
\citet{Chelton++:2000}. The TIWs had characteristic length and time scales for
these waves that were consistent with prior studies. But they also provided
clear evidence of TIWs south of the equator, and point out an evident (if
somewhat qualitative) asymmetry between the north and south SST profiles.
Although the SST signature of the southern TIWs was weaker, these waves were
able to persist for a few months longer than their northern counterparts. TIWs
were also observed in the Atlantic, but only along the northern equatorial
region. None were observed in the Indian Ocean.

We then sought to better understand the mechanism which produced these waves,
and presented the energetic analysis by \citet{Qiao-Weisberg:1998}. Advection
of PKE by the mean flow did not appear to play a major role, while the
barotropic generation of eddies by mean shear was especially significant. The
meridional shear between the SEC and EUC was observed to be the principal
source of PKE north of the equator, while the vertical and meridional SEC-EUC
shears south of the equator tended to consume PKE. There is also a modest
meridional advection of PKE by the eddies. The only significant role played by
PPE was to act as a reservoir for PKE; no baroclinic energy generation was
observed.

Although we have illustrated the importance of barotropic instability in TIW
dynamics, it should not be presumed that the waves are wholly due to such a
process. The study by Qiao and Weisberg doesn't, for example, address the
possibility of strong baroclinic processes near the cold tongue fronts, which
were outside the range of the TIWE experiment. Therefore, baroclinic theories
such as that suggested by \citet{Masina-Philander-Bush:1999} may yet provide an
accurate description of TIW dynamics. And yet, even though TIWs are associated
with strong SST gradients, \citet{Chelton++:2000} carefully make the point that
this is also the principal means of observing the waves, so it would be
difficult, if not impossible, to comment on their existence in the absence of
such gradients. It is also not clear how to coincide the lack of instability on
the southern side of the equator with the existence of southern TIWs. It is
clear however that it will be some time before a definitive explanation of TIW
dynamics is known.

\bibliography{tiw} \bibliographystyle{jas99}

\end{document}
