\documentclass[letterpaper, 11pt]{article}

\usepackage{amsmath, amssymb, amsfonts}
\usepackage{bm}
\usepackage[dvips]{graphicx}

\newcommand{\pdiff}[2]{\frac{\partial #1}{\partial #2}}

\title{Wake Angle}
\author{Marshall Ward}

\begin{document}

\maketitle

%%%%%%%%%%%%%%%%%%%%%%%%%%%%%%%%%%%%%%%%%%%%%%%%%%%%%%%%%%%%%%%%%%%%%%%%%%%%%%%

\section{Introduction}

Watch as I amazingly derive the fantastical Kelvin wake angles for geophysical systems.

%%%%%%%%%%%%%%%%%%%%%%%%%%%%%%%%%%%%%%%%%%%%%%%%%%%%%%%%%%%%%%%%%%%%%%%%%%%%%%%

\section{The Kelvin Wake Angle}

We first present a rederivation of the deep-water boat wake angle, as first described by Kelvin (ref). Although this result has been presented in more recent texts (Lighthill, Billingham+King), we shall stress some of the implicit assumptions that are involved and that must be reassessed when considering systems with different wave properties.

We seek to describe the stationary wake that is typically observed behind boats traveling at a roughly constant speed in a sufficiently deep body of water. The observed gravity waves satisfy a dispersion relation of the form
\begin{equation}
\omega = \sqrt{g k}
\end{equation}
so that the phase and group velocities are
\begin{equation}
c_p = \sqrt{\frac{g}{k}}, \qquad c_g = \frac{1}{2} \sqrt{\frac{g}{k}}.
\end{equation}
The relationship between the phase and group velocity is then $c_g = \frac{1}{2} c_p$.

The first thing we assume is that the boat (or some other source) produces an extremely broad spectrum of gravity waves, so that waves of all wavelengths populate the environment. A unique feature of the deep-water dispersion relation is that any wavelength $k$ can be produced, as long as the fluid is forced at the corresponding frequency $\omega(k)$. 

%%%%%%%%%%%%%%%%%%%%%%%%%%%%%%%%%%%%%%%%%%%%%%%%%%%%%%%%%%%%%%%%%%%%%%%%%%%%%%%

\section{The Shallow Water Wake Angle}

Fill me in. NOW!

%%%%%%%%%%%%%%%%%%%%%%%%%%%%%%%%%%%%%%%%%%%%%%%%%%%%%%%%%%%%%%%%%%%%%%%%%%%%%%%


\end{document}