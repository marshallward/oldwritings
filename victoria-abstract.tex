Experimental study of double diffusive gravity currents under rotation

Marshall Ward, Geophysical Fluid Dynamics Institute, Florida State University

George Veronis, Department of Geology and Geophysics, Yale University

We present an experimental study of two-layer double diffusive gravity currents with and without rotation. The roles of rotation, double diffusive tendency (e.g. a finger or diffusive layer favorable density profile), density difference, and solute concentration are considered. The currents are created by means of a lock-exchange experiment, where salt and sugar solutions of different densities and equal volume are separated by a barrier. We focus on systems with density differences no greater than $5 \times 10^{-4}$ g/cm$^{-3}$. Upon removal of the barrier, boundary currents similar to a Kelvin wave emerge, in the same manner as with a nondiffusive current. However, the double diffusive gravity currents are not only significantly faster than the equivalent nondiffusive flows, but also have a width that is considerably broader. In addition, features such as wall separation and formation of shear-like instabilities along the front are either much weaker or no longer present in the double diffusive cases. We also examine the evolution of the gravity current head speed. More data is required to confirm the interface displacement profile with time, but there is good evidence that, for rotating systems, head speeds are approximately piecewise constant, with relatively abrupt transitions in between. Speeds are also shown to be as sensitive to solute concentration as to small density differences.

\end